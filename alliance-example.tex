%% -----------------------------------------------------
%% Sample NSERC Alliance proposal template
%%
%% - Declare the document as "nserc-alliance",
%%   and specify whether the document is in French or English.
%% - Optionally, you can use the argument "nobullets" to hide all
%%   the instructions under each section title, and leave only
%%   your text.
%% - The pdf14 option can reduce PDF file size by not embedding
%%   standard fonts.
%% -----------------------------------------------------
\documentclass[french,
% nobullets, % uncomment to hide instructions
]{nserc-alliance}

%% --------------------------
%% Inclusion of a basic configuration file. Go see this file: there are
%% parameters (such as your name, etc.) to fill in. This avoids repeating
%% the same info in the case you produce multiple documents for the same
%% application.
%% --------------------------
%% -----------------------------------------------------
%% Configuration of an NSERC application
%% Defines a few macros that are global to all documents of the application
%% ----------------------------------------------------

%% If your application involves a company, put the name of the company
%% in a macro rather than writing it directly.
\newcommand{\namecompany}{Mr.\ Fusion}

%% Application year. Only appears in the metadata of the generated PDF.
\newcommand{\applicationyear}{2019}

%% The list of all authors of the application. Again, only useful for the
%% PDF metadata
\newcommand{\authorlist}{Emmett Brown}

%% The name and NSERC PIN of the main applicant
\nsercname{Emmett Brown}
\nsercpin{999999}

%% Documents are not dated
\date{}

%% Paragraphes français
%\setlength{\parindent}{0pt}

%% Hack to have list items displayed in a more compact way
\usepackage{paralist}
\setlength{\pltopsep}{4pt}
\setlength{\plitemsep}{4pt}

%% ----------
%% Loading a few packages. These are all optional and can be commented
%% out if you with. Feel free to add others.
%% ----------
\usepackage{hyperref}
\hypersetup{%
  pdfauthor = {\authorlist{}},
  pdfcreator = {NSERC Alliance LaTeX Template V1.0},
  pdfsubject = {NSERC \applicationyear{} \namecompany{}}
}
\usepackage{url}
\usepackage{todonotes}

%% ------------------------
%% Useful: a few "todo" macros to display colored boxes with remarks
%% and comments
%% ------------------------
\newcommand{\todoemmett}[1]{\todo[inline,caption={},color=cyan]{\sf\small Emmett: #1}}
\newcommand{\todomarty}[1]{\todo[inline,caption={},color=pink]{\sf\small Marty: #1}}
\newcommand{\todoall}[1]{\todo[inline,caption={},color=yellow]{\sf\small #1}}

%% ------------------------
%% Color for grayed out instruction bullets in the text. Change
%% this definition to show instructions with a different shade.
%% Comment it out to revert the instructions to black like the rest
%% of the text.
%% ------------------------
\definecolor{instructioncl}{gray}{0.35}

%% ------------------------
%% This will print a "DRAFT" watermark on all pages.
%% Uncomment the next two lines once the application is ready.
%% ------------------------
% \usepackage{draftwatermark}
% \SetWatermarkText{DRAFT}

%% Title for the proposal
\newcommand{\proposaltitle}{Super cool project title}

%% ----------
%% Set the title of the document in the PDF's metadata
%% ----------
\hypersetup{%
  pdftitle = {\proposaltitle{}}
}

\begin{document}
\thispagestyle{firstpage}
\maketitle

\noindent \textbf{Title: \proposaltitle}

%% -------------------------------
%% Background
%% -------------------------------
\section*{Background}
\ifinst\begin{instructions}
  \item Explain the challenge to be addressed, the importance of the topic and the need for new concepts or directions. 
  \item Outline the objectives of the project and briefly explain its anticipated outcomes and impact. 
  \item Position the proposed research relative to other efforts and to the state-of-the-art.
\end{instructions}\fi
  
\noindent Insert your text here, responding to each of the above points. 

You can also use references \cite{DBLPjournals/jsyml/Turing48,DBLPconf/afips/SolomonP76}.

%% -------------------------------
%% Partnership
%% -------------------------------
\section*{Partnership}
\ifinst\begin{instructions}
  \item List all partner organizations participating in the project. For each, describe their core activities and how they align with the project, their need for the proposed project, and their experience related to it, such as efforts to date to address the challenge.
  \item	Describe each partner organization's active role in the project, including defining the research questions, designing the research plan, collaborating or contributing to the research activities, co-supervising trainees and monitoring progress.
  \item Describe how the partner organizations will translate, mobilize and/or apply the research results to achieve the intended outcomes. 
  \item Explain the value and importance of each partner organization's involvement and other in-kind contributions to achieving the project's intended outcomes. If applicable, discuss how the combination of partner organizations is beneficial to the project. 
\end{instructions}\fi
  
\noindent Insert your text here, responding to each of the above points

%% -------------------------------
%% Proposal
%% -------------------------------
\section*{Research plan}
\ifinst\begin{instructions}
  \item Specify the research objectives and expected results. Describe the planned research activities, methodology and experimental design.
  \item Provide approximate timelines for the activities, milestones and deliverables. You may use a Gantt chart, table or diagram.
  \item Describe how equity, diversity and inclusion are considered in the research process (e.g., research questions, design, methodology, analysis, interpretation and dissemination of results) and how these considerations are integrated where relevant.
\end{instructions}\fi
  
\noindent Insert your text here, responding to each of the above points

%% -------------------------------
%% Team
%% -------------------------------
\section*{Team}
\ifinst\begin{instructions}
  \item List the applicant, any co-applicants, key participating staff of the partner organizations and any other key academic team members. For each, explain how their knowledge, expertise, experience and contributions align with the proposed project and describe their role in the project, as well as their roles and capabilities in training and mentoring trainees.
  \item	Briefly describe the plan for managing the project, along with the qualifications, roles and responsibilities of the team members involved in this respect.
\end{instructions}\fi
  
\noindent Insert your text here, responding to each of the above points

%% -------------------------------
%% Training plan
%% -------------------------------
\section*{Training Plan}
\ifinst\begin{instructions}
  \item Describe the learning experiences the project will provide, including the nature of interactions between trainees (undergraduate and graduate students, postdoctoral fellows) and partner organizations. 
  \item	Describe the research and professional skills that trainees will develop through these experiences and through their roles in the project.
  \item Explain how the research and professional skills gained by the trainees will prepare them for their future careers.
  \item Describe challenges to equity, diversity and inclusion in the context of your project's training environment and specify concrete practices you will implement to address them. You are encouraged to cite evidence supporting the proposed practices and to describe how you will monitor and adapt your actions based on non-demographic indicators of success.
\end{instructions}\fi
  
\noindent Insert your text here, responding to each of the above points

%% -------------------------------
%% Impact and benefits to Canada
%% -------------------------------
\section*{Impact and benefits to Canada}
\ifinst\begin{instructions}
  \item Explain how and the extent to which the proposed research will generate new knowledge in the natural sciences or engineering disciplines and/or develop or advance new technologies.
\item Considering the partner organizations' plans to use the research results, discuss how the project will lead to new or improved technologies, products, processes, services, policies, standards or regulations in Canada.
\item Describe how and the extent to which the project's intended outcomes will lead to economic, environmental, and/or other societal benefits to Canada and Canadians.
\end{instructions}\fi

\noindent Insert your text here, responding to each of the above points. 

%% -------------------------------
%% References
%% -------------------------------
\newpage
\begingroup
\section*{References}
\renewcommand{\section}[2]{}%
\ifinst\begin{instructions}
  \item Use this section to provide a list of the most relevant literature references. Do not refer readers to websites for additional information on your proposal. Do not introduce hyperlinks in your list of references.
  \item These pages are not included in the page count.
\end{instructions}\fi
\bibliographystyle{abbrv}
\bibliography{alliance-example}
\endgroup

\end{document}
%% :wrap=soft:folding=explicit: